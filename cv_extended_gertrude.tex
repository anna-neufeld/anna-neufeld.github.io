%%%%%%%%%%%%%%%%%%%%%%%%%%%%%%%%%%%%%%%%%
% Medium Length Graduate Curriculum Vitae
% LaTeX Template
% Version 1.1 (9/12/12)
%
% This template has been downloaded from:
% http://www.LaTeXTemplates.com
%
% Original author:
% Rensselaer Polytechnic Institute (http://www.rpi.edu/dept/arc/training/latex/resumes/)
%
% Important note:
% This template requires the res.cls file to be in the same directory as the
% .tex file. The res.cls file provides the resume style used for structuring the
% document.
%
%%%%%%%%%%%%%%%%%%%%%%%%%%%%%%%%%%%%%%%%%

%----------------------------------------------------------------------------------------
%	PACKAGES AND OTHER DOCUMENT CONFIGURATIONS
%----------------------------------------------------------------------------------------

\documentclass[margin, 10pt]{res} % Use the res.cls style, the font size can be changed to 11pt or 12pt here

\usepackage{helvet} % Default font is the helvetica postscript font
%\usepackage{newcent} % To change the default font to the new century schoolbook postscript font uncomment this line and comment the one above

\setlength{\textwidth}{5.1in} % Text width of the document
\usepackage{hyperref}

\title{Academic CV}

\begin{document}

%----------------------------------------------------------------------------------------
%	NAME AND ADDRESS SECTION
%----------------------------------------------------------------------------------------

\moveleft2.5\hoffset\centerline{\large\bf Anna Neufeld} % Your name at the top

%\section{\large Anna Neufeld}  %\hfill  %aneufeld@uw.edu

 
\moveleft\hoffset\vbox{\hrule width\resumewidth height 1pt}\smallskip % Horizontal line after name; adjust line thickness by changing the '1pt'
%\moveleft.5\hoffset\centerline{45 Lansing St Apt 2612} % Your address
%\moveleft.5\hoffset\centerline{San Francisco, San Francisco 94105}
%\moveleft.5\hoffset\centerline{(770) 401-6678}
%\moveleft.5\hoffset\centerline{josheppinette@icloud.com}

%----------------------------------------------------------------------------------------

\begin{resume}

%----------------------------------------------------------------------------------------
%	OBJECTIVE SECTION
%----------------------------------------------------------------------------------------
 

\section{Contact Information}
Department of Statistics \hfill  aneufeld@uw.edu\\
University of Washington \hfill \href{https://anna-neufeld.github.io}{https://anna-neufeld.github.io} \\
Padelford B-222  \hfill \\
%781-392-4894 \\
Seattle, WA, 98195



\section{Education} 

{\textbf{University of Washington},} Seattle, Washington. \hfill 2018-Present \\
Statistics PhD Student%, GPA 3.84 
%\begin{itemize}
%\item Advisor: Daniela Witten
%\item Selected coursework: advanced theory of statistical inference, advanced regression methods, measure theory, statistical machine learning, statistical demography, causal inference, spatial statistics, statistical analysis of networks
%\end{itemize}
%\begin{itemize}
 %   \item Relevant Coursework: Statistical Inference, Statistical Learning
%\end{itemize}

{\textbf{Williams College}}, Williamstown, MA \hfill 2014-2018  \\
BA in Mathematics (Highest Honors) and Computer Science. \\ 
\textit{Summa Cum Laude}, GPA 4.04. 
%\begin{itemize}
%\item Selected coursework: natural language processing, machine learning, artificial intelligence, theory of computation, algorithms, computer organization, programming languages, probability, regression and forecasting, real analysis, abstract algebra, dynamics of infectious disease, price and allocation theory, income distribution
%\end{itemize}
%\begin{itemize}
%\end{itemize}
 
%----------------------------------------------------------------------------------------
%	PROFESSIONAL EXPERIENCE SECTION
%----------------------------------------------------------------------------------------
 
\section{Research Experience}

{\textbf{Research Assistant (University of Washington)} \hfill 2020-Present
\begin{itemize}
\item Advisor: Daniela Witten
\item Developing a selective inference framework for constructing valid hypothesis tests and confidence intervals for within-region means or differences in means across splits in regression trees. 
\end{itemize} 

{\textbf{Williams College Undergraduate Honors Thesis}} \hfill 2017-2018 
\begin{itemize}
\item Advisor: Brianna Heggeseth
\item Proposed using longitudinal regression trees to study the impact of early exposure to environmental pollutants on the shape of body mass index trajectories later in life. After developing novel evaluation metrics to measure the ``shape-based" accuracy of a tree, developed a spline projection method that groups individuals by the shape of their individual trajectory while ignoring the level of their trajectory. 
\end{itemize} 

{\textbf{SMALL Research Experience for Undergraduates}} \hfill Summer 2016 
\begin{itemize}
\item Advisors: Julie Blackwood and Lauren Childs
\item Used compartmental differential equation models to quantify the relative contributions of sexual transmission and vector transmission in the spread of the Zika virus. 
\end{itemize} 


\section{Teaching Experience}
{\textbf{Visiting Lecturer}} \hfill Spring 2021 \\
Williams College Department of Computer Science
\begin{itemize}
\item Co-instructing CS 374T, Machine Learning, with Prof. Andrea Danyluk
\item Tutorial style course (modeled after Oxford University tutorials), which involves meeting with two students at a time. 10 total students enrolled in the course. 
\end{itemize}
{\textbf{Instructor of Record}} \hfill Summer 2019 \\
University of Washington Department of Statistics
\begin{itemize}
\item Stat 311: Elements of Statistical Methods
\item $\sim$60 students
\end{itemize}
{\textbf{Head Teaching Assistant}} \hfill Autumn 2019, Winter 2020 \\
University of Washington Department of Statistics
\begin{itemize}
\item Stat 311: Elements of Statistical Methods
\item Developed lab assignments, maintained lab website, helped write assignments and exams, served as liaison between professor and other TAs.  
\item $\sim$180 students total. 
\item Led two lab sections of 30 undergraduates each. 
\end{itemize} 
{\textbf{Graduate Teaching Assistant}} \hfill 2018-2020\\
University of Washington Department of Statistics
\begin{itemize}
\item Stat 311: Elements of Statistical Methods (Autumn 2018, Spring 2020)
\item Stat 423:  Applied Regression and Analysis of Variance (Winter 2019)
\item CSE/Stat 416: Introduction to Machine Learning (Spring 2019). 
\item Responsible for lab sections of around 30 undergraduates each. Also held office hours and graded assignments.
\end{itemize}
{\textbf{Undergraduate Teaching Assistant}} \hfill September 2015 - May 2018 \\
Williams College Departments of Computer Science, Mathematics, and Statistics
\begin{itemize}
\item Data Structures and Advanced Programming (Fall 2015), Linear Algebra (Spring 2016, Spring 2017), Abstract Algebra (Fall 2016), Regression and Forecasting (Fall 2017, Spring 2018)
\item Duties included grading homework, holding office hours, and running review sessions. 
\end{itemize}
{\textbf{Workshop Leader}} \hfill January 2018 \\
Williams College Office of Academic Resources
\begin{itemize}
\item Worked with a group of undergraduates and the Office of Academic Resources to pilot a new program of coding workshops
\item Taught a series of workshops in R to undergraduates from a variety of departments and graduate students from the Center for Development Economics. 
\end{itemize}
{\textbf{Peer Tutor}} \hfill 2016-2018 \\
Williams College Office of Academic Resources
\begin{itemize}
\item Nominated by faculty to serve as a peer tutor. Held one-on-one and drop-in tutoring sessions for microeconomics, macroeconomics, calculus, linear algebra, real analysis, statistics, and computer science.  
\item Also worked one on one with biology research students who needed help conducting data analysis in R. 
\end{itemize}

\section{Publications} 
\textbf{Neufeld, A.} and Witten, D. (2021). Discussion of Breiman’s “Two Cultures”: From Two Cultures to One. To appear in \emph{Observational Studies}.  \\
\\
Maxian, O*., \textbf{Neufeld, A.}*, Talis, E. J.*, Childs, L. M., \& Blackwood, J. C. (2017). Zika virus dynamics: When does sexual transmission matter?. Epidemics, 21, 48-55. \\
\small
(\emph{* denotes equal contribution})
\normalsize

%\section{In Preparation} 
%\textbf{Neufeld, A}, Gao, L.L, and Witten. \emph{Tree-Values: Selective Inference for Regression Trees}.

\section{Contributed Conference Presentations} 
Brianna Heggeseth* and \textbf{Anna Neufeld}.  Longitudinal Regression Trees: An Application to Environmental Exposure and Growth. In \emph{Joint Statistical Meetings}, Vancouver, BC, August 2018. \\
\\
Emma J. Talis*, Ondrej Maxian, and \textbf{Anna Neufeld}.  Understanding Zika Dynamics: Sex, Mosquitoes, and Gender. In \emph{Joint Mathematics Meetings}, Atlanta, GA, January 2017. \\
\\


\section{Software} 
\textbf{splinetree: longitudinal trees and forests using a spline projection method} \\
R package. Available from \href{https://github.com/anna-neufeld/splinetree}{github} and \href{https://cran.r-project.org/web/packages/splinetree/index.html}{CRAN}. 

\section{Professional Experience}
{\textbf{Cogo Labs, Cambridge, MA}} \hfill June 2017-August 2017 \\
Data Analytics Intern
\begin{itemize}
\item Worked with a team of engineers, designers, and analysts to build and market a website. Analyzed market data with SQL, analyzed site performance with google analytics and piwik, and assisted with backend web development in python. 
\end{itemize}

%\section{Talks and Posters} 
%\begin{itemize}
%\item Fake Williams Conference
%\item JMM
%\item The talks that Heggeseth put my name on? Find out if those talk. 
%end{itemize}


\section{Honors and Awards}
\textbf{University of Washington} \\
\begin{tabular}{l l l}
2021 & Nominated for Excellence in Teaching Award & \emph{Center for Teaching and Learning} \\
2020 & Dorothy M. Gilford Excellence in Teaching Award & \emph{Department of Statistics}
\end{tabular}  \\
\\
\textbf{Williams College} \\
\begin{tabular}{l l }
2018 & W. Marriott Canby Athletic Scholarship Prize   \\
& (for highest standing in scholarship among senior varsity athletes) \\
2018 & Robert M. Kozelka Prize in Statistics   \\
%& (presented to outstanding senior) \\
2018&  Sigma Xi Scientific Honor Society \\
2017 & Phi Beta Kappa  National Honor Society \\
%& (junior year inductee; top 5\% of class) \\
2016-2018 & New England Small Colleges Athletic Conference (NESCAC) All-Academic team \\
2016-2018 & Clare Boothe Luce Fellowship \\
2016 & Erastus C. Benedict First Prize in Mathematics \\
& (presented to outstanding sophomore) \\
\end{tabular}
%% sportsmanship
%5 nate kristof
%5 university wide 

\section{Mentoring}
{\textbf{University of Washington Undergrad Directed Reading Program (DRP)}}
\begin{itemize}
  \item Co-founded a program that pairs undergraduates with PhD student mentors for independent studies. Modeled after successful Directed Reading Programs (DRPs) in mathematics departments at several universities. More information can be found at \href{https://spa-drp.github.io}{our website}. 
  \item Served as a graduate student coordinator for 2020-2021. Have managed admissions, recruitment, and scheduling for over 30 undergraduate research projects. 
  \item Served as a mentor for the following undergraduate projects:
  \begin{itemize}
  \item Winter 2020, Christina Nick, Statistical Natural Language Processing
  \item Spring 2020, Rachael Ren, Infectious Disease Modeling with Differential Equations. See \href{https://spa-drp.github.io/writeups/spring2020/rachael.pdf}{project writeup} and \href{https://spa-drp.github.io/writeups/spring2020/rachaelslides.pdf}{presentation}.
  \item Autumn 2020, Harper Zhu, Infectious Disease Modeling on a Network.  See \href{https://spa-drp.github.io/writeups/aut2020/harper-slides.pdf}{presentation} and \href{https://harperzhu.shinyapps.io/DiseaseSimulation/}{shiny app}.
 \end{itemize}
 \end{itemize}



\section{Service}
\begin{itemize}
\item Reviewer for \textit{Statistical Science}.
 \item UW Statistics Graduate Student Representative (paid position that involves attending faculty meetings, serving on committees, and planning orientation for new students). 2020-2021.
    \item First round reviewer for PhD Admissions, 2020-2021. 
    \item UW Statistics Diversity, Inclusion, Community, and Equity (DICE) Committee, 2019-2021. Currently managing the new K-12 outreach subcommittee.
    \item Chair of the UW Statistics Fun Committee, 2019-2020.
    \item Organizer and Founder: \href{https://pearce790.github.io}{Statistics Education Reading Group}, 2019-2020.
  
   \end{itemize}
   
 \section{Other Activities}
\begin{itemize}
\item Competed for the varsity swim and dive team at Williams College from 2014-2018. Our team won the NESCAC (New England Small Colleges Athletic Association) championship and placed between 2nd and 4th nationally among NCAA Division 3 teams in each of these four years. As a senior, was selected by teammates and an alumni committee as the first female winner of the Nate Krissoff award, which recognizes purpose, dedication, and inspiration to fellow athletes.
\item Trip leader (2015), leader trainer (2016), and director of transportation and logistics (2017) for the Williams Outdoor Orientation for Living as First-years (WOOLF). 
  \item Williams College Great Ideas Committee (managed a budget of \$10,000 per year to implement improvements to undergraduate life), 2015-2017.
 \item Williams College Committee on Undergraduate Life (committee consisted of students, faculty, and staff), 2015-2016.

\end{itemize}
%{\textbf{Other}}
%\begin{itemize}

%\end{itemize}

%\section{Skills}
%\begin{itemize}
%\item Proficient: R, Python, Java, C, C++, matlab, Mathematica
%\item Familiar: SQL, html, javascript, CSS%, Scala, ML, Lisp
%\end{itemize}


}
\end{resume}
\end{document}